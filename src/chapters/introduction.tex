\end{multicols}
\chapter{Introduction}
\begin{multicols}{2}

\noindent Time series forecasting---predicting future values of variables within
temporal datasets---is largely an unsolved problem in complex or chaotic domains
such as weather (e.g. humidity, temperature or wind speed) and economics
(e.g. currency exchange rates or stock prices).  Examining the problem with the
latest models in the field of machine learning---Long Short-Term Memory
(\textsc{lstm}) based Recurrent Neural Networks (\textsc{rnn}s)---gives rise to
new hope of being able to predict time series in these domains.

\section{Background and Motivation}

Algorithmic trading combines the fields of finance and algorithmics/machine
learning. To understand the problem, a high level of knowledge is desirable in
both of these---and surrounding---domains.  Below, financial markets and
artificial neural networks as well as their connection to algorithmic trading is
briefly summarized.  It is further explained in the Theory chapter.

\subsection{Financial Markets}

The price movements on financial markets are difficult (even impossible,
according to the efficient-market hypothesis) to predict \citep{fama1995random}.
If one were able to predict the movements of, for example, stock prices or
currency exchange rates, that information could be used to ``beat the market''
(essentially, buying before a rise and selling before a drop) to continuously
increase the value of an investment portfolio, yielding a positive net return.

Predicting a financial market, whether the method used for prediction consists
of a fundamental- or technical approach, rests on devising a certain successful
strategy---looking at and considering a variety of factors and drawing a
conclusion as to how the market should be operated on (for example, deciding
at what point in time buying a certain financial asset is appropriate).

Furthermore, if the prediction process could be automated through the
application of an algorithm that decides when a financial asset should be bought
or sold with some level of accuracy (with respect to its predictions of price
movements)---in practice, letting the algorithm perform trading operations
autonomously on the market with real money---the algorithm would turn out to be
profitable, generating continuous positive net returns over time.

Many attempts have been made with some success. During the last decade, there
has been a rise in algorithmic trading from roughly 20 percent of the volume on
US equity markets in 2005 up to over 50 percent in 2010 \citep{kaya2016}, and
although the rise seems to have reached its limit during recent years, the sheer
volume of high-frequency trading implies some level of credibility for the
approach.

\subsection{Artificial Neural Networks}

Artificial neural networks (\textsc{ann}s) have been applied for several decades
to solve multinomial logistic regression (determining what \textit{class} a data
point belongs to) and linear regression (determining the \textit{value} of a
dependent variable as a function of a data point's independent variables)
problems.  The early versions of \textsc{ann}s (e.g.\ the single layer
perceptron) could only solve \textit{linearly separable} problems
\citep{rosenblatt1958}.  An \textsc{ann}, expressed in simple terms, is a
mathematical model of a biological brain, consisting of neurons, synapses and
dendrites, modeled by the \textsc{ann} through cells with scalars, activation
functions and weights.

The earlier models were also unable to predict time series (i.e.\ determining
what class a data point belongs to with respect to its movement over time, or
determining the value of a dependent variable as a function of the independent
variables' movement over time), something that was later solved through the
introduction of recurrent neural networks \citep{rumelhart1986}, which can
handle datasets with a temporal component dictating their intradependence and
distribution within the dataset, where each point in the sequence is correlated
to previous (with respect to time) points through several known and unknown
factors.

\textsc{rnn}s have historically had problems with vanishing (or exploding)
gradients---meaning they break down during training and become unable to model
the problem well \citep{pascanu2012}---but this problem was solved through the
introduction of the \textsc{lstm}-based \textsc{rnn}s \citep{hochreiter1997}.

Regardless of whether the problem is approached as a matter of classification or
regression, the temporal aspect renders algorithms and models not specifically
designed for the task impotent since one single input dataset could belong to
several different classes depending on where, in time or order, it appears in
the data sequence.  The problem should therefore be examined with algorithms
designed specifically for time series, such as \textsc{lstm}-based
\textsc{rnn}s.

Furthermore, \textit{sequence-to-sequence} learning is a method for training the
network on output sequences rather than providing it with a single target value
following the input sequence; instead of giving the model an input sequence and
the next, single value following the sequence, the model is given the
\textit{sequence} following the input sequence.  This is also the method used to
model the problem in this thesis and is explained in detail in the theory
chapter.

\section{Current Research}

Currently, in the field of machine learning, hidden Markov models
(\textsc{hmm}s), dynamic Bayesian networks (\textsc{dbn}s), tapped-delay neural
networks (\textsc{tdnn}s) and \textsc{rnn}s are commonly used to handle
sequential datasets, and have been applied with some degree of success to
financial markets \citep{saad1998,kita2012,zhang2004}.

Although much research is currently being done into the application of
\textsc{lstm}-based \textsc{rnn}s in financial markets, the research is still in
its infancy, and although the \textsc{lstm}-based \textsc{rnn} has been around
for twenty years, it is only recently that progress is being made to a
significant degree in practice.  The volume of data created in the past couple
of years---about 90 percent of all data ever created by humans
\citep{devakunchari2014}---together with relatively cheap but very fast graphics
processing units (\textsc{gpu}s), have opened up the field of machine learning
to more cost efficient research.

More recently, deep \textsc{lstm}-based \textsc{rnn}s have been applied with
some level of success in predicting future time series from historical
sequential datasets; such as predicting the weather for the next twenty-four
hours \citep{zaytar2016}, or predicting the next sequence of words in natural
language \citep{sutskever2014sequence}.

While natural language has obvious structure (i.e. \textit{grammar}), weather,
instead, is seemlingy chaotic, controlled by many unknown factors.  Despite
this; considering the success in weather prediction, \textsc{lstm}-based
\textsc{rnn}s may have the ability to take into account these factors,
especially given an appropriate set of \textit{features}---input data processed
to make certain aspects more prominent, thus enabling the algorithm to model the
problem more easily.

\section{Problem Statement}

Financial time series are used ubiquitously in algorithmic trading.  In
algorithmic trading, it is imperative that accurate predictions are made about
numerous variables (such as volatility and returns) in order to time market
entry and exit.

Since deep \textsc{lstm}-based \textsc{rnn}s (specifically using
sequence-to-sequence learning) have not been applied within the algorithmic
trading domain before, and since they have shown success in solving similar
problems in other domains, it raises the question whether the technique can be
used to predict a future sequence of financial variables that can be used to
time both entry and exit positions within a certain time horizon.  The
uniqueness of this approach is the prediction of \textit{sequences} at a time,
rather than predicting a single data point on each iteration.

Assuming that correlations exist along the temporal dimension of the dataset,
the problem is reduced to a matter of finding an appropriate set of features
enhancing the correlations, on which to train the \textsc{lstm}-based
\textsc{rnn}.  Expressed in a more concise manner, we
attempt to answer the question: \\

--- Can price movements on financial markets be modeled through the application
of \textsc{lstm}-based \textsc{rnn}s, specifically using
\textit{sequence-to-sequence} learning, and if so, how does the
\textsc{lstm}-based \textsc{rnn} compare to the traditional \textsc{rnn}.
