Det neurala nätverket implementerades i Python (Källa till python?) 2.7 med biblioteket Tensorflow (källa dit). Det neurala nätverket var ett Recurrent Neural Network \textit{(RNN)} med Long Short Term Memory \textit{(LSTM)}. Specifika hyperparametrar ändrades under arbetets gång för att optimera nätverket. 

Studien är avgränsad mot valutahandel då högfrekvens-data över aktier har en hög efterfrågan. Det gör att den är svår, men framförallt dyr, att få tag på. Datan som användes hämtades från \textit{FOREX} och visar högfrekvenshandel med valutor. 

FOREX data över valutahandel var det som det neurala nätverket fick träna på och sedan predicera den antingen uppgång, nedgång eller stillestånd för en valutas värde. Detta jämfördes sedan med korrekt data och en slutsats kunde lätt dras angående modellens pricksäckerhet. 

Det gjordes även jämförelser mot andra "state-of-the-art"-algoritmer för att ha möjlighet att utvärdera effektivititen av nätverket. 






Metod
* Forskningsmetod
* Forskningstrategi
* Valda Tekniker
* Datainsamling
* Dataanalysmetod

\textbf{vilken empiri behövs?} Tror inte

\textbf{måste vi prata med folk?} Tror inte

\textbf{måste vi läsa dokument?} Tror inte

\textbf{måste vi titta på det vi ska studera?} Ja?

\textbf{måste vi delta i något sammanhang?} Nej

\textbf{design av studien}

\textbf{vilken metod ska användas?}
\textit{Så in i helvetes kvantitativ}

\textbf{varför denna metod?}
\textit{Vad annars?*}

\textbf{var ska fältarbetet göras?}
\textit{At my battlestation}

\textbf{hur ska empiri samlas?}
\textit{Hacking forex}

\textbf{hur ska empiri tolkas?}
\textit{Maskininlärning, närmare bestämt RNN med LSTM}

\textbf{vilka alternativa metoder?}
\textit{Andra maskininlärnings-algoritmer?}

\textbf{metodreflektion}

\textbf{vilka fördelar respekteive nackdelar medför valda metoder?}
\textit{Fördelar vet vi inte, därför vi ska testa det?}

\textbf{hur har ni planerat att öka tillförlitlighet resp validitet reliabilitet?}
\textit{WTF does she mean?}

\textbf{design av studiens genomförande ska vara så detaljerad som möjligt en tumregel är att insamling och bearbetning av empiri och data ska utgöra 20\% av uppsatsen}
