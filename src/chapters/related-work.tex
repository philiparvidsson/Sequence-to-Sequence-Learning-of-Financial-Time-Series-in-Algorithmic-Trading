\end{multicols}
\chapter{Related Work}
\begin{multicols}{2}

\noindent As much as we were inspired by related work, we also encountered the
problem of breaking new ground during our research.  Especially in regard to the
application of the \textsc{lstm}-based \textsc{rnn} on financial markets using
sequence-to-sequence learning, finding related work turned out to be somewhat
difficult.

\section{Financial Market Predictions}

Enormous amounts of research has been made into the prediction of financial
markets, and we have waded through a large volume of it during the thesis work.
We found \citet{vahala2016} particularly helpful in approaching the problem of
financial market prediction using the methods in this thesis.

When it comes to feature extraction, we did not have the time to test all
features that we felt were applicable to the problem.  A very interesting and
inspiring piece of work in regard to feature extraction is \citet{Chan201305}.

\section{Sequence-to-Sequence Learning}

The bulk volume of research on sequence-to-sequence learning has been done on
machine translation.  Although algorithmic trading is an entirely different
problem, the model configuration is very similar in its setup.  The theory
behind sequence-to-sequence learning with an \textsc{lstm} model is covered by
\citet{sutskever2014sequence}, wherein the practical application of
sequence-to-sequence learning is explained as well.

\citet{chollet2015}---specifically the Keras
documentation---provides a deep insight into the intricacies of
sequence-to-sequence learning and its practical applications.

Although modeling the stochastic processes on financial markets is a different
(and perhaps more complex) problem compared to modeling long-term weather,
\citet{zaytar2016} provides a deeper insight into the theory of the
\textsc{lstm}-based \textsc{rnn}, detailing it further than what was done in
this thesis, as well was providing a solid foundation for understanding the use
of the \textsc{lstm} in modeling and predicting stochastic processes.
